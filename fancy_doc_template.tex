% My template for document with custom margins, a nice looking titlepage and 
% header & footer.
% Jack-Benny Persson <jack-benny@cyberinfo.se>.

\documentclass[11pt,a4paper]{article} 
\usepackage[table]{xcolor}
\usepackage[a4paper]{geometry}
\geometry{verbose,tmargin=2.5cm,bmargin=2.5cm,lmargin=2.54cm,rmargin=2.54cm}
\usepackage{fancyhdr}
\usepackage{lastpage}

\newcommand{\HRule}{\rule{\linewidth}{0.5mm}}

\pagestyle{fancy}
\fancyhf{}
\chead{Wheather report}
\cfoot{Page \thepage\ of \pageref{LastPage}}



\begin{document}

\begin{titlepage}
\begin{center}
\vspace*{3.5 cm}
\textsc{\LARGE Wheather report}\\[1.5cm]


\HRule \\[0.5cm]
{ \huge \bfseries Week 18}\\[0.4cm]
\HRule \\[1.5cm]

\textsc{\Large Draft 1}\\[0.5cm]
\textsc{\Large 2012-05-05}\\[0.5cm]

\end{center}
\end{titlepage}

\tableofcontents
\pagebreak

\section{Introduction}

A sample \LaTeX\ document to demonstrate. Can be used as a template.


\subsection{Nice wheater all week long}

All week we'll have really nice wheater with lots of sunshine and warm 
temperature for the month.
Some thunder may occur in the afternoons.
And that concludes the summary of the weather.

\subsubsection{Thunder may occur}

As we said earlier, some thunder may occur.

And in this new paragraph, well yada yada yada
And som bla bla bla.

\subsection{Blah blah blah}

Yada yada yada.

\textit{Blah blah blah?}


\section{Some mathematics}

We can also show of some math in \LaTeX\

\(2x+1=5\) is \(x=2\)

And some more, like this \(x+x=y\)


\pagebreak

\section{Tables}
Here I will show a simple table.

\begin{table}[h]
\begin{center}
    \begin{tabular}{ | l | l | l | p{7cm} |}
    \hline
    \cellcolor[gray]{0.9}\textbf{Day} & 
    \cellcolor[gray]{0.9}\textbf{Min Temp} & 
    \cellcolor[gray]{0.9}\textbf{Max Temp} & 
    \cellcolor[gray]{0.9}\textbf{Summary} \\ \hline
    Monday & 12C & 23C & Sunshine all day long.  \\ \hline
    Tuesday & 15C & 20C & Some clouds, other than that nice weahter. \\ \hline
    Wednesday & 10C & 20C & Some rain and possibly thunder during the evening. 
        \\ \hline
    \end{tabular}
\end{center}
\end{table}
And thats the weather report for the week. Nice huh? In a table and everything 
with the top row in gray and all.
\end{document}

